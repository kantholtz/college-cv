\documentclass{report}

% \usepackage[ngerman]{babel}
\usepackage[utf8]{inputenc}

% fonts
\usepackage{geometry,amsmath,amsfonts,metalogo,hyperref,mdwlist,array,multicol,fontawesome,color}
\usepackage[default,osf]{sourcesanspro}
\usepackage[scaled=.95]{sourcecodepro}
\linespread{1.3}

% figures
\usepackage{subfig}
\usepackage{float}
\restylefloat{figure}

\usepackage{wrapfig}

\begin{document}


%
%   SHOW OFF
%


{\centering

  \begin{figure}
    \vspace{3cm}

    \centering
    \includegraphics[width=0.3\textwidth]{src/yield}

    \vspace{4cm}
  \end{figure}


  {\Huge\textbf{Yield Sign Detection}}
  \vspace{.4cm}

  Felix Hamann

  \vspace{.2cm}

  \today

}


%
%   CONTENT
%

\tableofcontents


\chapter{Introduction}

This document describes the implementation of a system to detect yield
signs from images.

{\color{red}{
    \begin{enumerate}
    \item \textit{Notation definitions}
    \item \textit{Image test set}
    \item \textit{Requirements}
      \begin{enumerate}
      \item \textit{Changing light (night, day, dawn, dusk)}
      \item \textit{Performance}
      \end{enumerate}
    \end{enumerate}
}}


\pagebreak
\section{Filter pipeline overview}
...

\begin{enumerate}
\item Binarization

  \begin{enumerate}
  \item Red Amplification
  \item Binarization
  \end{enumerate}

\end{enumerate}


\pagebreak
\section{Binarization}

The first step of the whole pipeline consists of extracting the red
color from the coloured input image \textit{source}. Thus a mapping
for each pixel from three dimensions to one dimension is
needed. Simply returning the red color component does not suffice, as
with larger values of the green and blue components the source color
either shifts towards yellow, magenta or white. A very simple method
that is used in this application takes the red component, optionally
amplifies it by some factor and substracts the green and blue
components values.

Let \( \Gamma = \{0, 1, ..., 255\} \) be all possible pixel values, \(
\delta \in \Gamma \) the threshold and \( \alpha \in \mathbb{R},
\alpha \geq 1 \) the factor for red amplification, then the pixel
written to the binarized image \textit{target} is:

\begin{equation}\label{eq:binarization}
  \begin{split}
    f & : \Gamma \times \Gamma \times \Gamma \to \mathbb{Z} \\
    f(r, g, b) & = \alpha r - (g + b) \\
    target(y, x) & =
    \begin{cases}
      255 & \quad \text{if } f(source(y, x)) > \delta \\
      0   & \quad \text{else}
    \end{cases}
  \end{split}
\end{equation}

This method has one drawback however. There is no distinction between
less saturated or darker shades of red and orange or magenta. This is
a result of not taking the distance between green and blue into
account. An example is depicted in Table
\ref{table:binarization}. More elaborate variations of this
calculation were tested (considering the ratio of red to all colors or
weighting by green-blue distance) but this did not increase the
overall quality of red detection. Actually, without constantly white
balancing the camera used for capturing, narrowing the range of red
would perform worse as the ambient light differs greatly over the
course of a day. The drawback of this approach are more falsely marked
areas of the image that have to be considered in all further steps.

\begin{table}
  \begin{tabular}{c r r r r r}

    \textbf{Color} & \textbf{Red} & \textbf{Green} & \textbf{Blue} &
    \textbf{\textit{f(r, g, b)}} & \textbf{\textit{target(y, x)}} \\ \hline

    \parbox[c]{1em}{\includegraphics[width=1em]{src/bin_000000}} &   0 &   0 &   0 &    0 &   0 \\ \hline
    \parbox[c]{1em}{\includegraphics[width=1em]{src/bin_ffffff}} & 255 & 255 & 255 &  -51 &   0 \\ \hline
    \parbox[c]{1em}{\includegraphics[width=1em]{src/bin_00ff00}} &   0 & 255 &   0 & -255 &   0 \\ \hline
    \parbox[c]{1em}{\includegraphics[width=1em]{src/bin_0000ff}} &   0 &   0 & 255 & -255 &   0 \\ \hline
    \parbox[c]{1em}{\includegraphics[width=1em]{src/bin_ff0000}} & 255 &   0 &   0 &  459 & 255 \\ \hline
    \parbox[c]{1em}{\includegraphics[width=1em]{src/bin_ff8888}} & 255 & 125 & 125 &  209 & 255 \\ \hline
    \parbox[c]{1em}{\includegraphics[width=1em]{src/bin_ff8800}} & 255 & 125 &   0 &  334 & 255 \\ \hline
    \parbox[c]{1em}{\includegraphics[width=1em]{src/bin_ff0088}} & 255 &   0 & 126 &  334 & 255

  \end{tabular}
  \caption{Example results of function \textit{f} in
    \ref{eq:binarization} with \( \alpha=1.8, \delta=200 \). Note that
    the result is equivalent for light red, orange and magenta.}
  \label{table:binarization}
\end{table}

Performance is nice.


\subsection{Foo}

bar

\end{document}
